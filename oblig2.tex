\documentclass[a4paper, english, 11pt]{article}
\usepackage[utf8]{inputenc}
\usepackage[T1]{fontenc}
\usepackage{babel,graphicx,mathpple,textcomp,varioref}
\usepackage{amssymb, amsmath}
\usepackage{mathtools}
\usepackage{epstopdf}
\usepackage{ulem}
\usepackage{listings}
\usepackage{color}
\usepackage{subfig}
\usepackage{listings}

\usepackage{hyperref}
\hypersetup{
    colorlinks,
    citecolor=black,
    filecolor=black,
    linkcolor=black,
    urlcolor=black
}

\setlength{\parindent}{0.0in}
\setlength{\parskip}{0.1in}

\DeclareGraphicsRule{.tif}{png}{.png}{`convert #1 `dirname #1`/`basename #1 .tif`.png}

\definecolor{listinggray}{gray}{0.9}
\definecolor{lbcolor}{rgb}{0.9,0.9,0.9}

\lstset{
    backgroundcolor=\color{lbcolor},
    tabsize=4,
    rulecolor=,
    language=python,
    basicstyle=\scriptsize,
    upquote=true,
    aboveskip={1.5\baselineskip},
    columns=fixed,
    showstringspaces=false,
    extendedchars=true,
    breaklines=true,
    prebreak = \raisebox{0ex}[0ex][0ex]{\ensuremath{\hookleftarrow}},
    frame=single,
    showtabs=false,
    showspaces=false,
    showstringspaces=false,
    identifierstyle=\ttfamily,
    keywordstyle=\color[rgb]{0,0,1},
    commentstyle=\color[rgb]{0.133,0.545,0.133},
    stringstyle=\color[rgb]{0.627,0.126,0.941},
}


\title{INF5390 oblig 2}
\author{mathiajj and martinvl}

\begin{document}
\maketitle

\section*{Task 1}

\subsection*{Domain:}

\subsubsection*{Constants:}

The domain does not have any known constants.

\subsubsection*{Functions:}

The domain has the following functions:

\begin{itemize}

	\item
	canfool(X, Y) - Returns the set of times when X can fool Y

\end{itemize}

\subsubsection*{Relations:}

The domain has the following relations:

\begin{itemize}

	\item
	Man(X) - (X) exists if X is a man
	\item
	Woman(X) - (X) exists if X is a woman
	\item
	Vegetarian(X) - (X) exists if X is a vegetarian
	\item
	Smart(X) - (X) exists if X is smart
	\item
	Politician(X) - (X) exists if X is a politician
	\item
	Barber(X) - (X) exists if X is a barber
	\item
	Hate(X, Y) - (X, Y) exists if X hate Y
	\item
	Like(X, Y) - (X, Y) exists if X like Y
	\item
	Shaves(X, Y) - (X, Y) exists if X shaves Y

\end{itemize}

All relations and functions has a fixed arity, but we use the paranthesis to
define arguments as it is easier to read.

\subsection*{a.}

$\forall X \forall Y ((Vegetarian(Y) \wedge  Hates(X, Y)) \Rightarrow
Smart(X))$

\subsection*{b.}

$\forall X \forall Y (Smart(X) \wedge Vegetarian(X) \Rightarrow (\neg
Like(Y, X))$

\subsection*{c.}

$\exists X \forall Y ((Woman(X) \wedge (Man(Y) \wedge Vegetarian(Y))
\Rightarrow Like(X, Y))$

\subsection*{d.}

$\exists X \forall Y ((Barber(X) \wedge Man(Y) \wedge \neg Shaves(Y, Y))
\Rightarrow Shaves(X, Y))$

\subsection*{e.}

$\forall X(Politican(X) \Rightarrow ((\exists Y \forall T
(T \in canfool(X, Y)) \vee \forall Y \exists T (T \in canfool(X, Y))) \wedge
\exists Y \exists T (T \not\in canfool(X, Y))))$

\section*{Task 2}

$\forall X \forall Y \exists T ((Sock(X) \wedge Sock(Y) \wedge Pair(X, Y))
\Rightarrow (Before(Now, T) \wedge (Lost(X, T) \vee Lost(Y, T)))$

\section*{Task 3}
\subsection*{a.}
State space is simply the space of all possible states. When talking about plan
space we typically mean the space of all possible sequences of actions that
may or may not lead from a given initial state to a given goal state.

\subsection*{b.}
Progressive planning uses the initial state as a starting point and then tries
to reach the goal state by applying possible actions (i.e. actions for which
the preconditions satisfied). Regressive planning starts with the goal state,
backtracking the actions that can lead to the current state in an attempt to
find the initial state.

\subsection*{c.}
Commitment is relevant in partial order planning. Least commitment means that
you leave as much of the plan ordering undefined as possible (while leaving the
plan valid). More commitment means to order more of the actions than what is
necessary, where the extreme is a totally ordered plan.

\subsection*{d.}
A totally ordered plan is essentially a clearly defined sequence of actions.
In a partially ordered plan the actions are ordered with relative links ("A
must happen before B"). This means that parts of a partially ordered plan can
be executed in any order (as long as the constraints in the plan are
satisfied). It also means that several totally ordered plans might satisfy a
single partially ordered plan.

\subsection*{Schema instances}
\begin{align*}
    &\mathrm{Fly}(P_1, \mathrm{JFK}, \mathrm{SFO})\\
    &\mathrm{Fly}(P_2, \mathrm{SFO}, \mathrm{JFK})\\
    &\mathrm{Fly}(P_1, \mathrm{JFK}, \mathrm{JFK})\\
    &\mathrm{Fly}(P_2, \mathrm{SFO}, \mathrm{SFO})
\end{align*}

\end{document}
